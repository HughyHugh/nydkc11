\documentclass[hidelinks]{article}

\usepackage[margin=1in]{geometry}
\usepackage[colorlinks=true, linkcolor=black, urlcolor=blue]{hyperref}

\title{\texttt{https://nydkc11.org} Maintainance Manual}
\author{Brian Lee}
\date{April 2018}

\newcommand{\snippet}[1]{{\centering\texttt{#1}\par}}
\newcommand{\tilda}{{\raise.17ex\hbox{$\scriptstyle\mathtt{\sim}$}}}
\begin{document}

\maketitle

\section{Setup \& Installation}

\subsection{Pre-Requesites}

The following must be set up before you can begin developing on the website GitHub repository:

\begin{itemize}
\item \textbf{A working UNIX-type OS.} You might be able to get by with Mac OS, but the website code works most optimally with Linux. 
This is because some of the website's dependencies require Linux.
My recommendation is to install the \href{https://www.ubuntu.com/}{Ubuntu} as a dualboot with whatever existing system you're using.\footnote{If you're already using Linux, use that instead.}
\item \textbf{A working installation of Python 3}. This DOES NOT mean Python 2. Don't worry about specific package installation for now -- \texttt{pip} will handle that for you using \texttt{requirements.txt} on the repository.
You should also have a intermediate -- advanced knowedge of Python.
\item \textbf{A separate virtual environment}. You should keep all Python website-related assets (e.g. \texttt{pip} installations) in a virtual environment specific to it.
\item \textbf{A working text editor you are familiar with}. My recommendation is \texttt{vim}.
\item \textbf{A working GitHub account}. You should also have been granted contributer access by your predecessor.
\item \textbf{A working installation of} \texttt{git}, and a \textit{proficient} understanding of using \texttt{git}.
\end{itemize}

If you don't know what any of these requirements are, please research them in-depth and ensure you do before beginning development.
I \textbf{strongly} encourage you to go through Django's \href{https://docs.djangoproject.com/en/1.11/intro/tutorial01/}{excellent tutorial} first before touching any website code.

\subsection{Installation}

Please follow this step-by-step guide for setting up your development platform.

\begin{enumerate}
\item Activate the virtual environment you've setup for this project, and change into a directory you deem appropriate to house the project.
As a reminder, a working virtual environment is a pre-requesite to this project. 
Your terminal header should look something like this:

\snippet{(env-name) account@host:\tilda{}/path/to/desired/project/directory}

\item Clone the website's GitHub repository. As of April 2018, the command

\snippet{git clone https://github.com/nydkcd11/nydkc11.git}

should suffice, but you should check for the specific URL on the repository. 

\item \texttt{cd} into the newly generated folder \texttt{nydkc11} and install the necessary packages par \texttt{requirements.txt}. 
This is accomplishable with

\snippet{pip install -r requirements.txt}

\textbf{Please check that your virtual environment is setup before running this command}.

\item \texttt{cd} into the inner \texttt{nydkcd11} folder,\footnote{Please ignore the naming discrepancy.} and re-generate the tables for the database. 

\snippet{python manage.py migrate}

You should have a basic understanding of the basic \texttt{makemigrations} $\to$ \texttt{migrate} workflow for making changes to the website's database.
If you do not, please visit Django's documentation.

\item Run the test development server with \texttt{python manage.py runserver}. 
Enter \texttt{http://127.0.0.1:8000/} into your web browser, and you should see a (very minimal) version of the website.
It should appear that way because you haven't put any test assets into the website yet.
Be sure that all the components of the website that should be functional are working before you start developing.

\end{enumerate}

\end{document}
